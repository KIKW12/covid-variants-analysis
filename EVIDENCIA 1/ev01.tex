% Options for packages loaded elsewhere
\PassOptionsToPackage{unicode}{hyperref}
\PassOptionsToPackage{hyphens}{url}
%
\documentclass[
]{article}
\usepackage{amsmath,amssymb}
\usepackage{iftex}
\ifPDFTeX
  \usepackage[T1]{fontenc}
  \usepackage[utf8]{inputenc}
  \usepackage{textcomp} % provide euro and other symbols
\else % if luatex or xetex
  \usepackage{unicode-math} % this also loads fontspec
  \defaultfontfeatures{Scale=MatchLowercase}
  \defaultfontfeatures[\rmfamily]{Ligatures=TeX,Scale=1}
\fi
\usepackage{lmodern}
\ifPDFTeX\else
  % xetex/luatex font selection
\fi
% Use upquote if available, for straight quotes in verbatim environments
\IfFileExists{upquote.sty}{\usepackage{upquote}}{}
\IfFileExists{microtype.sty}{% use microtype if available
  \usepackage[]{microtype}
  \UseMicrotypeSet[protrusion]{basicmath} % disable protrusion for tt fonts
}{}
\makeatletter
\@ifundefined{KOMAClassName}{% if non-KOMA class
  \IfFileExists{parskip.sty}{%
    \usepackage{parskip}
  }{% else
    \setlength{\parindent}{0pt}
    \setlength{\parskip}{6pt plus 2pt minus 1pt}}
}{% if KOMA class
  \KOMAoptions{parskip=half}}
\makeatother
\usepackage{xcolor}
\usepackage[margin=1in]{geometry}
\usepackage{color}
\usepackage{fancyvrb}
\newcommand{\VerbBar}{|}
\newcommand{\VERB}{\Verb[commandchars=\\\{\}]}
\DefineVerbatimEnvironment{Highlighting}{Verbatim}{commandchars=\\\{\}}
% Add ',fontsize=\small' for more characters per line
\usepackage{framed}
\definecolor{shadecolor}{RGB}{248,248,248}
\newenvironment{Shaded}{\begin{snugshade}}{\end{snugshade}}
\newcommand{\AlertTok}[1]{\textcolor[rgb]{0.94,0.16,0.16}{#1}}
\newcommand{\AnnotationTok}[1]{\textcolor[rgb]{0.56,0.35,0.01}{\textbf{\textit{#1}}}}
\newcommand{\AttributeTok}[1]{\textcolor[rgb]{0.13,0.29,0.53}{#1}}
\newcommand{\BaseNTok}[1]{\textcolor[rgb]{0.00,0.00,0.81}{#1}}
\newcommand{\BuiltInTok}[1]{#1}
\newcommand{\CharTok}[1]{\textcolor[rgb]{0.31,0.60,0.02}{#1}}
\newcommand{\CommentTok}[1]{\textcolor[rgb]{0.56,0.35,0.01}{\textit{#1}}}
\newcommand{\CommentVarTok}[1]{\textcolor[rgb]{0.56,0.35,0.01}{\textbf{\textit{#1}}}}
\newcommand{\ConstantTok}[1]{\textcolor[rgb]{0.56,0.35,0.01}{#1}}
\newcommand{\ControlFlowTok}[1]{\textcolor[rgb]{0.13,0.29,0.53}{\textbf{#1}}}
\newcommand{\DataTypeTok}[1]{\textcolor[rgb]{0.13,0.29,0.53}{#1}}
\newcommand{\DecValTok}[1]{\textcolor[rgb]{0.00,0.00,0.81}{#1}}
\newcommand{\DocumentationTok}[1]{\textcolor[rgb]{0.56,0.35,0.01}{\textbf{\textit{#1}}}}
\newcommand{\ErrorTok}[1]{\textcolor[rgb]{0.64,0.00,0.00}{\textbf{#1}}}
\newcommand{\ExtensionTok}[1]{#1}
\newcommand{\FloatTok}[1]{\textcolor[rgb]{0.00,0.00,0.81}{#1}}
\newcommand{\FunctionTok}[1]{\textcolor[rgb]{0.13,0.29,0.53}{\textbf{#1}}}
\newcommand{\ImportTok}[1]{#1}
\newcommand{\InformationTok}[1]{\textcolor[rgb]{0.56,0.35,0.01}{\textbf{\textit{#1}}}}
\newcommand{\KeywordTok}[1]{\textcolor[rgb]{0.13,0.29,0.53}{\textbf{#1}}}
\newcommand{\NormalTok}[1]{#1}
\newcommand{\OperatorTok}[1]{\textcolor[rgb]{0.81,0.36,0.00}{\textbf{#1}}}
\newcommand{\OtherTok}[1]{\textcolor[rgb]{0.56,0.35,0.01}{#1}}
\newcommand{\PreprocessorTok}[1]{\textcolor[rgb]{0.56,0.35,0.01}{\textit{#1}}}
\newcommand{\RegionMarkerTok}[1]{#1}
\newcommand{\SpecialCharTok}[1]{\textcolor[rgb]{0.81,0.36,0.00}{\textbf{#1}}}
\newcommand{\SpecialStringTok}[1]{\textcolor[rgb]{0.31,0.60,0.02}{#1}}
\newcommand{\StringTok}[1]{\textcolor[rgb]{0.31,0.60,0.02}{#1}}
\newcommand{\VariableTok}[1]{\textcolor[rgb]{0.00,0.00,0.00}{#1}}
\newcommand{\VerbatimStringTok}[1]{\textcolor[rgb]{0.31,0.60,0.02}{#1}}
\newcommand{\WarningTok}[1]{\textcolor[rgb]{0.56,0.35,0.01}{\textbf{\textit{#1}}}}
\usepackage{graphicx}
\makeatletter
\def\maxwidth{\ifdim\Gin@nat@width>\linewidth\linewidth\else\Gin@nat@width\fi}
\def\maxheight{\ifdim\Gin@nat@height>\textheight\textheight\else\Gin@nat@height\fi}
\makeatother
% Scale images if necessary, so that they will not overflow the page
% margins by default, and it is still possible to overwrite the defaults
% using explicit options in \includegraphics[width, height, ...]{}
\setkeys{Gin}{width=\maxwidth,height=\maxheight,keepaspectratio}
% Set default figure placement to htbp
\makeatletter
\def\fps@figure{htbp}
\makeatother
\setlength{\emergencystretch}{3em} % prevent overfull lines
\providecommand{\tightlist}{%
  \setlength{\itemsep}{0pt}\setlength{\parskip}{0pt}}
\setcounter{secnumdepth}{-\maxdimen} % remove section numbering
\ifLuaTeX
  \usepackage{selnolig}  % disable illegal ligatures
\fi
\IfFileExists{bookmark.sty}{\usepackage{bookmark}}{\usepackage{hyperref}}
\IfFileExists{xurl.sty}{\usepackage{xurl}}{} % add URL line breaks if available
\urlstyle{same}
\hypersetup{
  pdftitle={Evidencia 1 \textbar{} Análisis Inicial},
  pdfauthor={Juan Enrique Ayala Zapata-A01711235},
  hidelinks,
  pdfcreator={LaTeX via pandoc}}

\title{Evidencia 1 \textbar{} Análisis Inicial}
\author{Juan Enrique Ayala Zapata-A01711235}
\date{2024-04-26}

\begin{document}
\maketitle

\hypertarget{parte-1-responde-justificadamente-las-siguientes-preguntas}{%
\section{Parte 1: Responde justificadamente las siguientes
preguntas}\label{parte-1-responde-justificadamente-las-siguientes-preguntas}}

\begin{enumerate}
\def\labelenumi{\arabic{enumi}.}
\tightlist
\item
  Investiga, ¿Cuál es la situación actual de COVID-19 a nivel mundial,
  en México, en tu estado de origen y en tu municipio, alcaldía o
  colonia? De acuerdo al sitio de la Johns Hopkins University \&
  Medicine, dejó de colectar información sobre el COVID-19 el 10 de
  marzo de 2023, debido a que dejó de ser considerada como pandemia,
  pero de acuerdo a la última medición se tiene la información de que
  había 676,609,955 casos activos; con 6,881,955 muertos registrados en
  el mundo.
\end{enumerate}

En México, a la última fecha de la colección de datos por el Johns
Hopkins se tiene un total de 7,483,444 casos; habiendo 333,188 muertes
confirmadas hasta la última fecha.

De acuerdo a la Secretaría de Salud de Michoacán, a la última fecha de
actualización que es el 29 de junio de 2023, en toda la entidad
federativa se contaron un total hasta la fehca de 221,611 casos
confirmados; habiendo un total contado hasta la fecha de 8,83
defunciones.

Así mismo, en la entidad de Lázaro Cárdenas, hubo un total de 22,165
casos confirmados hasta la última fecha de corte, sin defunciones
reportadas en la misma fecha.

\begin{enumerate}
\def\labelenumi{\arabic{enumi}.}
\setcounter{enumi}{1}
\item
  ¿Cuál fue la primera variante del virus que se propagó a todo el
  mundo? De acuerdo al sitio nextstrain.org/ncov/global, la primera
  variante que se propagó a todo el mundo fue la Wuhan/WH04/2024.
\item
  ¿Cuáles son las otras variantes del virus que existen en otras
  regiones del mundo? De acuerdo al National Library of Medicine y la
  OMS (a fecha de 13 de julio de 2021), existen 10 variantes de
  preocupación y de interés:
\end{enumerate}

Variantes de preocupación (Linaje de Pangolín): Alpha (B.1.1.7)
Beta(B.1.351) Gamma(P.1) Delta(B.1.617.2)

Variantes de interés (Linaje de Pangolín): Epsilon(B.1.427/B.1.429)
Zeta(P.2) Eta(B.1.525) Theta(P.3) Iota(B.1.526) Kappa(B.1.617.1)

\begin{enumerate}
\def\labelenumi{\arabic{enumi}.}
\setcounter{enumi}{3}
\item
  ¿Cómo buscarías información de la variante del virus en tu país? Para
  obtener la información necesaria sobre la variante del virus en mi
  país, tengo que dirigirme a la página oficial de lsa Secretaría de
  Salud, para así obtener la infotmación más veraz sobre el virus en
  México.
\item
  Imagina que te encuentras en una situación similar a la de Li
  Wenliang, médico chino que intentó alertar sobre el brote de
  coronavirus en su país, pero fue detenido por las autoridades y
  obligado a retractarse, ¿qué harías en su caso? Selecciona un
  inciso:a) Lo reportas al centro de investigación o la universidad. b)
  Lo reportas a la prensa. c) Guardas la información. Como persona en la
  misma situación que Li Wenliang, yo reportaría el caso a un centro de
  investigación o a universidades para que trabajaran de manera pronta
  en encontrar la cura y/o vacuna para la enfermedad, para así prevenir
  la mayor cantidad de muertes. Así mismo, como médico, me vería en la
  oblicación ética de reportarlo, debido a que con esto se lograría
  mitigar el riesgo de propagación de la enfermedad.
\end{enumerate}

\hypertarget{parte-2-realiza-lo-siguiente-en-un-script-de-r-y-cuxf3digo}{%
\section{Parte 2: Realiza lo siguiente en un script de R y
código:}\label{parte-2-realiza-lo-siguiente-en-un-script-de-r-y-cuxf3digo}}

\hypertarget{cargar-librerias}{%
\subsection{Cargar librerias}\label{cargar-librerias}}

\begin{Shaded}
\begin{Highlighting}[]
\FunctionTok{library}\NormalTok{(stringr)}
\FunctionTok{library}\NormalTok{(Biostrings)}
\end{Highlighting}
\end{Shaded}

\begin{verbatim}
## Loading required package: BiocGenerics
\end{verbatim}

\begin{verbatim}
## 
## Attaching package: 'BiocGenerics'
\end{verbatim}

\begin{verbatim}
## The following objects are masked from 'package:stats':
## 
##     IQR, mad, sd, var, xtabs
\end{verbatim}

\begin{verbatim}
## The following objects are masked from 'package:base':
## 
##     anyDuplicated, aperm, append, as.data.frame, basename, cbind,
##     colnames, dirname, do.call, duplicated, eval, evalq, Filter, Find,
##     get, grep, grepl, intersect, is.unsorted, lapply, Map, mapply,
##     match, mget, order, paste, pmax, pmax.int, pmin, pmin.int,
##     Position, rank, rbind, Reduce, rownames, sapply, setdiff, sort,
##     table, tapply, union, unique, unsplit, which.max, which.min
\end{verbatim}

\begin{verbatim}
## Loading required package: S4Vectors
\end{verbatim}

\begin{verbatim}
## Loading required package: stats4
\end{verbatim}

\begin{verbatim}
## 
## Attaching package: 'S4Vectors'
\end{verbatim}

\begin{verbatim}
## The following object is masked from 'package:utils':
## 
##     findMatches
\end{verbatim}

\begin{verbatim}
## The following objects are masked from 'package:base':
## 
##     expand.grid, I, unname
\end{verbatim}

\begin{verbatim}
## Loading required package: IRanges
\end{verbatim}

\begin{verbatim}
## 
## Attaching package: 'IRanges'
\end{verbatim}

\begin{verbatim}
## The following object is masked from 'package:grDevices':
## 
##     windows
\end{verbatim}

\begin{verbatim}
## Loading required package: XVector
\end{verbatim}

\begin{verbatim}
## Loading required package: GenomeInfoDb
\end{verbatim}

\begin{verbatim}
## 
## Attaching package: 'Biostrings'
\end{verbatim}

\begin{verbatim}
## The following object is masked from 'package:base':
## 
##     strsplit
\end{verbatim}

\begin{Shaded}
\begin{Highlighting}[]
\FunctionTok{library}\NormalTok{(ggplot2)}
\end{Highlighting}
\end{Shaded}

\hypertarget{asigna-a-un-vector-los-nombres-de-los-diferentes-archivos}{%
\subsection{Asigna a un vector los nombres de los diferentes
archivos}\label{asigna-a-un-vector-los-nombres-de-los-diferentes-archivos}}

\begin{Shaded}
\begin{Highlighting}[]
\NormalTok{variantes }\OtherTok{\textless{}{-}} \FunctionTok{c}\NormalTok{(}\StringTok{"wuhan.fna"}\NormalTok{, }\StringTok{"ALPHA.fasta"}\NormalTok{, }\StringTok{"beta.fasta"}\NormalTok{, }\StringTok{"GAMMA.fasta"}\NormalTok{, }\StringTok{"Omnicron.fasta"}\NormalTok{)}
\end{Highlighting}
\end{Shaded}

\hypertarget{asigna-a-un-vector-los-nombres-de-las-variantes}{%
\subsection{Asigna a un vector los nombres de las
variantes}\label{asigna-a-un-vector-los-nombres-de-las-variantes}}

\begin{Shaded}
\begin{Highlighting}[]
\NormalTok{nombre\_variantes }\OtherTok{\textless{}{-}} \FunctionTok{c}\NormalTok{(}\StringTok{"Wuhan"}\NormalTok{, }\StringTok{"Alpha"}\NormalTok{, }\StringTok{"Beta"}\NormalTok{, }\StringTok{"Gamma"}\NormalTok{, }\StringTok{"Omicron"}\NormalTok{)}
\end{Highlighting}
\end{Shaded}

\hypertarget{funcion-para-contar-las-bases-de-adn}{%
\subsection{Funcion para contar las bases de
ADN}\label{funcion-para-contar-las-bases-de-adn}}

\begin{Shaded}
\begin{Highlighting}[]
\NormalTok{contar\_bases }\OtherTok{\textless{}{-}} \ControlFlowTok{function}\NormalTok{(ADN) \{}
\NormalTok{  a\_count }\OtherTok{\textless{}{-}} \FunctionTok{str\_count}\NormalTok{(ADN, }\AttributeTok{pattern =} \StringTok{"A"}\NormalTok{)}
\NormalTok{  t\_count }\OtherTok{\textless{}{-}} \FunctionTok{str\_count}\NormalTok{(ADN, }\AttributeTok{pattern =} \StringTok{"T"}\NormalTok{)}
\NormalTok{  g\_count }\OtherTok{\textless{}{-}} \FunctionTok{str\_count}\NormalTok{(ADN, }\AttributeTok{pattern =} \StringTok{"G"}\NormalTok{)}
\NormalTok{  c\_count }\OtherTok{\textless{}{-}} \FunctionTok{str\_count}\NormalTok{(ADN, }\AttributeTok{pattern =} \StringTok{"C"}\NormalTok{)}
  \FunctionTok{data.frame}\NormalTok{(}\AttributeTok{Base =} \FunctionTok{c}\NormalTok{(}\StringTok{"A"}\NormalTok{, }\StringTok{"T"}\NormalTok{, }\StringTok{"G"}\NormalTok{, }\StringTok{"C"}\NormalTok{), }\AttributeTok{Count =} \FunctionTok{c}\NormalTok{(a\_count, t\_count, g\_count, c\_count))}
\NormalTok{\}}
\end{Highlighting}
\end{Shaded}

\hypertarget{funciuxf3n-para-calcular-gc}{%
\subsection{Función para calcular
\%GC}\label{funciuxf3n-para-calcular-gc}}

\begin{Shaded}
\begin{Highlighting}[]
\NormalTok{adn\_info }\OtherTok{\textless{}{-}} \ControlFlowTok{function}\NormalTok{(ADN) \{}
\NormalTok{  adn\_len }\OtherTok{\textless{}{-}} \FunctionTok{nchar}\NormalTok{(ADN)}
\NormalTok{  gc\_count }\OtherTok{\textless{}{-}} \FunctionTok{str\_count}\NormalTok{(ADN, }\AttributeTok{pattern =} \StringTok{"G|C"}\NormalTok{)}
\NormalTok{  gc\_percentage }\OtherTok{\textless{}{-}}\NormalTok{ (gc\_count }\SpecialCharTok{/}\NormalTok{ adn\_len) }\SpecialCharTok{*} \DecValTok{100}
  \FunctionTok{cat}\NormalTok{(}\StringTok{"Tamaño de las bases:"}\NormalTok{, adn\_len, }\StringTok{"}\SpecialCharTok{\textbackslash{}n}\StringTok{"}\NormalTok{)}
  \FunctionTok{cat}\NormalTok{(}\StringTok{"Porcentaje de GC:"}\NormalTok{, gc\_percentage, }\StringTok{"\%}\SpecialCharTok{\textbackslash{}n\textbackslash{}n}\StringTok{"}\NormalTok{)}
\NormalTok{\}}
\end{Highlighting}
\end{Shaded}

\hypertarget{funcion-para-generar-secuencia-contrasentido}{%
\section{Funcion para generar secuencia
contrasentido}\label{funcion-para-generar-secuencia-contrasentido}}

\begin{Shaded}
\begin{Highlighting}[]
\NormalTok{generar\_contrasentido }\OtherTok{\textless{}{-}} \ControlFlowTok{function}\NormalTok{(ADN\_set) \{}
\NormalTok{  contrasentido }\OtherTok{\textless{}{-}} \FunctionTok{reverseComplement}\NormalTok{(ADN\_set)}
  \FunctionTok{return}\NormalTok{(contrasentido)}
\NormalTok{\}}
\end{Highlighting}
\end{Shaded}

\hypertarget{main}{%
\subsection{Main}\label{main}}

\begin{Shaded}
\begin{Highlighting}[]
\CommentTok{\# Main}
\ControlFlowTok{for}\NormalTok{ (i }\ControlFlowTok{in} \DecValTok{1}\SpecialCharTok{:}\FunctionTok{length}\NormalTok{(variantes)) \{}
  \CommentTok{\# Imprime el nombre de la variante}
  \FunctionTok{cat}\NormalTok{(}\StringTok{"Nombre de la variante:"}\NormalTok{, nombre\_variantes[i], }\StringTok{"}\SpecialCharTok{\textbackslash{}n}\StringTok{"}\NormalTok{)}
  
  \CommentTok{\# Lee el archivo de ADN y cuenta las bases}
\NormalTok{  ADN\_set }\OtherTok{\textless{}{-}} \FunctionTok{readDNAStringSet}\NormalTok{(variantes[i])}
\NormalTok{  adn }\OtherTok{\textless{}{-}} \FunctionTok{toString}\NormalTok{(ADN\_set)}
\NormalTok{  base\_counts }\OtherTok{\textless{}{-}} \FunctionTok{contar\_bases}\NormalTok{(adn)}
  
  \CommentTok{\# Gráfico de barras para las bases de ADN}
  \FunctionTok{ggplot}\NormalTok{(base\_counts, }\FunctionTok{aes}\NormalTok{(}\AttributeTok{x =}\NormalTok{ Base, }\AttributeTok{y =}\NormalTok{ Count, }\AttributeTok{fill =}\NormalTok{ Base)) }\SpecialCharTok{+}
    \FunctionTok{geom\_bar}\NormalTok{(}\AttributeTok{stat =} \StringTok{"identity"}\NormalTok{) }\SpecialCharTok{+}
    \FunctionTok{labs}\NormalTok{(}\AttributeTok{title =} \FunctionTok{paste}\NormalTok{(}\StringTok{"Cantidad de bases nitrogenadas de"}\NormalTok{, nombre\_variantes[i]),}
         \AttributeTok{x =} \StringTok{"Base Nitrogenada"}\NormalTok{, }\AttributeTok{y =} \StringTok{"Count"}\NormalTok{) }\SpecialCharTok{+}
    \FunctionTok{theme\_minimal}\NormalTok{()}
  
  \CommentTok{\# Mostrar el gráfico}
  \FunctionTok{print}\NormalTok{(}\FunctionTok{ggplot}\NormalTok{(base\_counts, }\FunctionTok{aes}\NormalTok{(}\AttributeTok{x =}\NormalTok{ Base, }\AttributeTok{y =}\NormalTok{ Count, }\AttributeTok{fill =}\NormalTok{ Base)) }\SpecialCharTok{+}
          \FunctionTok{geom\_bar}\NormalTok{(}\AttributeTok{stat =} \StringTok{"identity"}\NormalTok{) }\SpecialCharTok{+}
          \FunctionTok{labs}\NormalTok{(}\AttributeTok{title =} \FunctionTok{paste}\NormalTok{(}\StringTok{"Cantidad de bases nitrogenadas de"}\NormalTok{, nombre\_variantes[i]),}
               \AttributeTok{x =} \StringTok{"Base Nitrogenada"}\NormalTok{, }\AttributeTok{y =} \StringTok{"Count"}\NormalTok{) }\SpecialCharTok{+}
          \FunctionTok{theme\_minimal}\NormalTok{())}
  
  \CommentTok{\# Calcula y muestra el \%GC}
  \FunctionTok{adn\_info}\NormalTok{(adn)}
  
  \CommentTok{\# Genera la secuencia contrasentido y almacénala en una variable}
\NormalTok{  contrasentido }\OtherTok{\textless{}{-}} \FunctionTok{generar\_contrasentido}\NormalTok{(ADN\_set)}
  \CommentTok{\# Ahora puedes usar la variable \textquotesingle{}contrasentido\textquotesingle{} según sea necesario}
\NormalTok{\}}
\end{Highlighting}
\end{Shaded}

\begin{verbatim}
## Nombre de la variante: Wuhan
\end{verbatim}

\includegraphics{ev01_files/figure-latex/unnamed-chunk-7-1.pdf}

\begin{verbatim}
## Tamaño de las bases: 29903 
## Porcentaje de GC: 37.97278 %
## 
## Nombre de la variante: Alpha
\end{verbatim}

\includegraphics{ev01_files/figure-latex/unnamed-chunk-7-2.pdf}

\begin{verbatim}
## Tamaño de las bases: 29884 
## Porcentaje de GC: 37.83295 %
## 
## Nombre de la variante: Beta
\end{verbatim}

\includegraphics{ev01_files/figure-latex/unnamed-chunk-7-3.pdf}

\begin{verbatim}
## Tamaño de las bases: 3244622 
## Porcentaje de GC: 37.27405 %
## 
## Nombre de la variante: Gamma
\end{verbatim}

\includegraphics{ev01_files/figure-latex/unnamed-chunk-7-4.pdf}

\begin{verbatim}
## Tamaño de las bases: 4699287 
## Porcentaje de GC: 36.52482 %
## 
## Nombre de la variante: Omicron
\end{verbatim}

\includegraphics{ev01_files/figure-latex/unnamed-chunk-7-5.pdf}

\begin{verbatim}
## Tamaño de las bases: 208393 
## Porcentaje de GC: 37.88371 %
\end{verbatim}

\hypertarget{interpretaciuxf3n-de-gruxe1ficas}{%
\section{Interpretación de
gráficas}\label{interpretaciuxf3n-de-gruxe1ficas}}

Las gráficas muestran la cantidad de bases nitrogrenadas presentes en
cada cadena de ADN de cada variente de Covid-19 seleccionada para el
estudio durante este proyecto, se puede observar que las variantes.

Por ejemplo, al hacer una comparación de las bases nitrogenadas para la
variante de Wuhan y la variante Alpha, podemos observar una ligera
disminución de Adenida y Citocina.

ASí mismo, gracias a estas gráficas podemos observar la variación de la
cantidad de Bases Nitrogenadas del ADN de cada variante, notando así que
en algunas variantes superan las 60,000.

\hypertarget{referencias}{%
\section{Referencias}\label{referencias}}

COVID-19 Map - Johns Hopkins Coronavirus Resource Center. (s.~f.). Johns
Hopkins Coronavirus Resource Center.
\url{https://coronavirus.jhu.edu/map.html}

De Comunicación Social, C. G. (s.~f.). Informe diario COVID-19
Michoacán. Michoacán.
\url{https://salud.michoacan.gob.mx/informe-diario-covid-19-michoacan-190/}

Variantes del SARS-COV-2, una historia todavia inacabada-PMC. (2021, 13
julio). National Library Of Medicine. Recuperado 24 de abril de 2024, de
\url{https://www.ncbi.nlm.nih.gov/pmc/articles/PMC8275477/}

Genomic Epidemiology of Novel Coronavirus. (s.~f.). Nextrain.org.
\url{https://web.archive.org/web/20200420123520/https://nextstrain.org/ncov/global}

\end{document}
